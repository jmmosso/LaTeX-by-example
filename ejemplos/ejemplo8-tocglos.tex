\documentclass{book}

%JMANUEL
\usepackage[utf8]{inputenc}  %con el formato de applemac no andan los acentos directos.
\usepackage[spanish]{babel}

\usepackage[spanish]{translator}
\usepackage[toc,acronym]{glossaries}
\makeglossaries
\newglossaryentry{computadora}
{
  name=computadora,
  description={se debe considerar que esencialmente es un aparato que hay en muchas casas que sirve para programar y divertirse.}
}

\newglossaryentry{lavarropas}
{
  name=lavarropas,
  description={se debe considerar que esencialmente es un aparato que hay en muchas casas que sirve para lavar la ropa.}
}

\newacronym{pdf}{PDF}{Portable Document Format}
\newacronym{ids}{IDS}{Intrusion Detection Systems}




\begin{document}

\title{TOC con glosario y acróninos}
\author{Juan Manuel Ramon}
\date{Setiembre de 2012}
\maketitle

\frontmatter

% Para agregar el índice, puede ser omitido.
\tableofcontents


\chapter{Introducción}
Aquí escribo la introducción. Cada párrafo se separa con una línea en blanco.
\mainmatter
 
\chapter{Primero lo primero}
Puedo hacer que el texto vaya en cursiva con \emph{texto en cursiva}. Hacer una enumeración:
\begin{enumerate}
\item Linux
\item OpenBSD
\item FreweBSD
\end{enumerate}

La \gls{computadora} no tiene nada que ver con el \gls{ids}, y mucho menos con el \gls{lavarropas}.
Toda la documentación de los \gls{ids} deberá estar en formato \gls{pdf}.

\chapter{Segundo le sigue}
Las notas a  pie se hacen con\footnote{Texto que aparecerá en la nota a pie de página.}.

\section{Introduccción}
Texto de la sección de introducción...
La \gls{computadora} no tiene nada que ver con el \gls{ids}, y mucho menos con el \gls{lavarropas}.
Toda la documentación de los \gls{ids} deberá estar en formato \gls{pdf}.

\section{Estructura}
Texto de la sección de etructuras...

\subsection{Primera subsección}
Texto de la priimera subsección de estructuras...

\subsection{Segunda subsección}
Texto de la segunda subsección de estructuras...

\appendix
\chapter{Primer apéndice}
Texto del priimer apéndice...
La \gls{computadora} no tiene nada que ver con el \gls{ids}, y mucho menos con el \gls{lavarropas}.
Toda la documentación de los \gls{ids} deberá estar en formato \gls{pdf}.

\chapter{Segundo apéndice}
Texto del segundo apéndice...

\backmatter

\newpage
\printglossary[type=main]
\newpage
\printglossary[type=\acronymtype]



\end{document}
