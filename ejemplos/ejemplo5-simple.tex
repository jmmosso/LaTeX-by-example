\documentclass{article}
%\documentclass[10pt,a4paper,twoside]{book}

%JMANUEL
\usepackage[utf8]{inputenc}  %con el formato de applemac no andan los acentos directos.
\usepackage[spanish]{babel}
\usepackage{graphicx}

\begin{document}

Existen dos formas de generar bibliografías en documentos, una simple \cite{simple} por medio de directivas Latex que puede ser utilizada para artículos cortos en las que no se utilizan gran cantidad de referencias, y otra mas compleja \cite{bibtex} para Tesis o Libros en cuyo caso se utiliza una herramienta suplementaria conocida como "bibTex" .

\begin{thebibliography}{1}

  \bibitem{simple} John W. Dower {\em Readings compiled for History
  21.479.}  1991.

  \bibitem{bibtex}  The Japan Reader {\em Imperial Japan 1800-1945} 1973:
  Random House N.Y. 

  \end{thebibliography}

\end{document}
